\documentclass[12]{article}

\usepackage[utf8]{inputenc}
\usepackage[T1]{fontenc}
\usepackage{graphicx}
\usepackage[ngerman]{babel}
\usepackage{amsmath}
\usepackage{parskip}
\usepackage{enumitem}
\usepackage{bm}
\usepackage{color}

\usepackage{fancyhdr}
\usepackage{titlesec}
\usepackage[a4paper,left=25mm,right=25mm,top=25mm,bottom=30mm]{geometry}

\pagestyle{empty}
\pagestyle{fancyplain}
\headheight 35pt
\lhead{Gruppe: Lara Croft\\}
\chead{\textbf{\Large One Sheet}}
\rhead{Rapid Game Development\\\today}
\lfoot{}
\cfoot{}
\rfoot{\small\thepage}
\titleformat*{\section}{\large\bfseries}
\setlength{\parindent}{4em}

\begin{document}

\section*{Titel: \textit{Skateboard 2D}} 

\section*{Alter der Spieler} 
	3 - 90 Jahre
	
\section*{Outline} 
	Als Skateboarder flüchtet man vor Gefängniswärtern, macht Stunts und
versucht die Level so schnell wie möglich zu durchfahren. Dabei skatet man durch
mystische Wälder, Felder Straßen und Fabriken. Durch perfekte Landungen und
antreten des Skateboardes baut man Geschwindigkeit auf, man kann aber auch
stürzen, falls man Stunts falsch ausführt.


\section*{Konkurrenzprodukte}
	Olli Olli World, Tony Hawk usw.
	
\section*{USK} ab 0

\section*{Unique Selling Points (USP)}
	\begin{itemize}
		\item unverbrauchtes Game Design (wenige Skateboardspiele)
		\item Spricht die Fans der Tony Hawk Pro Skater Reihe an.
		\item sehr schneller Gameflow
		\item kompetitiver Aspekt durch Bestenlisten
	\end{itemize}
	Ziel-Plattformen: PC mit Smartphone-Spiel Flair.
\end{document}
