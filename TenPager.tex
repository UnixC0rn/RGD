\documentclass[paper=a4,fontsize=12pt,ngerman]{scrartcl}

\usepackage[utf8]{inputenc} 
\usepackage[T1]{fontenc}
\usepackage{graphicx}
\usepackage[ngerman]{babel}
\usepackage{amsmath}
\usepackage[a4paper,left=25mm,right=25mm,top=25mm,bottom=30mm]{geometry}
\usepackage{parskip}
\usepackage{url}
% 

\begin{document}

\pagestyle{plain}

% Einbinden der Titelseite
\begin{titlepage}

\linespread{1.5}

\begin{center}
	 \large
	 \hfill
	 \vfill
	 \Large{\bfseries{Skateboard 2D}}
 		
 	for Windows \\
 	 
	\normalsize
 	\vfill
	 Age: 3 - Up \\
	 USK: 0 \\
	 Ship Date: 31.09.2021
	 
 	\vfill
	\vfill

	Désirée Becker, Marilyn Konrad, Jan Klasen, Lorenz Wollstein, Jana Koch, Laszlo Maximowitsch, Arne Kreuz
	\\ Saarbruecken, den 17.07.2020
\end{center}

\end{titlepage}


\clearpage

% Erstes Kapitel
\section*{Game Outline:}
Skateboard 2D ist ein Jump \& Run, in dem man mit einem selbst erstellten Charakter von
links nach rechts durch Level skatet, begleitet von passender, schneller Musik.
Es kann zwischen mehreren Hintergrundgeschichten gewählt werden, sei es als
entkommener Gefängnisinsasse oder die Flucht von einem misslungenen Date. Die
einzelnen Level bringen den Spieler durch mystische Wälder, über Straßen und Felder,
welche alle eigene Eigenschaften haben (siehe Kapitel Spielewelt \& Charakter).
Durch diese Auswahlmöglichkeiten werden verschiedene Spielweisen ermöglicht, je
nachdem welchen Spielstil man präferiert.
In den Leveln selbst kann der Spieler springen, Stunts vollführen, beschleunigen und, je
nach Spielstil, das gewünschte Ziel mit diesen Mitteln erreichen. Stunts und die folgende
Landung erhöhen die Geschwindigkeit, falls sie korrekt ausgeführt wurden, was natürlich
sehr wichtig für Speedruns und Ähnliches ist.
Falls man dabei einen Fehler macht, verlangsamt sich die Geschwindigkeit, oder der
Charakter stürzt sogar vom Skateboard, was die Geschwindigkeit wieder komplett
zurücksetzt.
Das Ganze wird natürlich auch in entsprechenden Bestenlisten festgehalten, in denen man
sich eventuell mit anderen Skatern messen kann.

\clearpage

% Zweites Kapitel
\section*{Mechaniken:}
Grundsätzlich bewegt sich der Spieler im klassischen Jump \& Run Stil von links nach rechts.
Der Spieler startet bei der Geschwindigkeit 0 auf dem Skateboard und dieses beschleunigt
von alleine, bis zu einem bestimmten Maximum.
Dieses Tempo soll in etwa etwas mehr als Schrittgeschwindigkeit darstellen, durch
verschiedene Interaktionen soll der Spieler die Möglichkeit haben dieses Tempo zu erhöhen.
\\ \\
Diese Interaktionen (auch Button-Press) sind wie folgt beschrieben:
\begin{itemize}
	\item \textbf{Antreten:} \\ Durch das Antreten auf dem Boden kann der Spieler mit Hilfe seines Fußes das Skateboard
	beschleunigen.
	Die Geschwindigkeit die hiermit erreicht werden kann, kann ein bestimmtes Maximum nicht
	überschreiten (falls man schneller sein sollte, gibt das Antreten keinen zusätzlichen Effekt).
	Durch jeden Button-Press bekommt man einen festen Boost welcher zur Geschwindigkeit
	dazu addiert wird. Die Boosts haben einen Cool-Down damit man ihn nicht spamen kann.
	Werden keine 'perfect landings' ausgeführt und/ oder nicht weiter angetreten, sinkt die
	Geschwindigkeit nach und nach wieder auf den Standard ab.
	
	\item \textbf{Sprung:} \\ Der Spieler kann mit einem Tastendruck einen Sprung durchführen. Dieser dient zur
	Höhengewinnung. Der Sprung hat immer die gleiche Höhe. Die zurückgelegte Entfernung
	eines Sprungs wird durch die Geschwindigkeit definiert.
	
	\item \textbf{Stunt:} \\ Der Spieler kann (ausschließlich) in der Luft einen 'Stunt' ausführen. Hierfür muss er in der
	Luft einen Button drücken.
	\begin{itemize}
		\item Idee 1: Jeder Stunt gibt die gleiche Anzahl von Punkten und die Zeit ist immer gleich.
		Lediglich die animation ist random.
		\item Idee 2: Es gibt mehrere Stunts, welche verschieden lange andauern und eine
		unterschiedliche Anzahl von Punkten geben.
	\end{itemize}
	Jeder Stunt hat eine feste Zeitspanne. Während dieser Zeit darf der Spieler keine Kollision
	mit anderen Dingen haben (zB. Boden). Sollte dies dennoch passieren, fällt der Spieler vom
	Skateboard und die Geschwindigkeit wird auf null zurückgesetzt.
	Es sind feste Spawnpunkte definiert um zu verhindern, dass der Spieler zB. in einem Hindernis spawned.
	
	\item \textbf{Perfect landing:} (schwierigere Mechanik)\\ Beim 'perfect landing' erhält der Spieler die Möglichkeit seine maximale Geschwindigkeit zu
	überschreiten.
	Durch einen Tastendruck, welcher innerhalb einer bestimmten Zeit nach der Landung
	getätigt werden muss, erhält der Spieler einen zusätzlichen Geschwindigkeits-Boost.
	Dieser zusätzliche Boost baut sich nach und nach wieder ab. Durch häufiges Aneinanderreihen mehrerer 'perfect landings' bevor diese sich abgebaut haben kann die Geschwindigkeit
	auf ein Extremes erhöht werden. (Hier ist das Limit sehr hoch angesetzt)
	Ein 'perfect landing' kann man sich sozusagen als sauberes Abfangen des Fallens vorstellen
	wodurch das Momentum von dem Fall in Geschwindigkeit umgewandelt wird.
	Sollte der Spieler zu früh versuchen ein 'perfect landing' auszuführen, dann fällt er vom
	Skateboard und startet wieder bei Geschwindigkeit 0.
	Sollte der Spieler zu spät versuchen ein 'perfect landing' zu probieren, erhält man keinen Boost
	und alle anderen 'perfect landing'-Boosts werden entfernt.
	Ein 'perfect landing' gibt nur einen Boost, wenn es nach einem erfolgreichen Stunt ausgeführt wird.
\end{itemize}
	
	Die Zeit des Runs wird gemessen und zum Schluss angezeigt. Durch die verschiedenen
	Mechaniken wird gewährleistet, dass ein Spieler ein Level in sehr vielen
	unterschiedlichen Zeiten abschließen kann. Dadurch wird die Wettbewerbsfähigkeit des
	Spiels ausgebaut.\\
	
	Außerdem gibt es einen dauerhaften Verfolger, welcher versucht den Spieler zu erreichen.
	Wenn der Spieler zu langsam ist wird er von diesem gefangen und es ist ein Game Over.
	Der Verfolger ist deaktivierbar.
\clearpage

\section{Charakter:}
Zu Beginn kann man im Menü seinen Charakter aus ein paar gegebenen Kleidungsstücken,
Haaren, Schuhen und Figurmodellen(Geschlecht, Hautfarbe \& evtl. Größe) selbst erstellen.
Zu Beginn kann zwischen zwei Skateboards ausgewählt werden, neue
Skins und bessere Skateboards werden im jeweils höheren Level bzw. durch besondere
Tricks freigeschaltet. Alter des Charakters soll Anfang/Mitte 20 sein und der Name kann vom
Spieler selbst ausgesucht werden (Eingabe des Namens nach
Erstellung des Charakters oder evtl. auch davor, was sich eher ergibt).
Die Hintergrundgeschichte ist ebenfalls auswählbar. Man kann aus drei vorgegebenen
Storys auswählen:
\begin{enumerate}
	\item Charakter ist nach langer Planung gerade aus dem Gefängnis ausgebrochen
	und auf der Flucht. Zufälligerweise findet er auf dem Parkplatz des Gefängnisses ein
	Skateboard und nimmt dieses als 'Fluchtfahrzeug'. (evtl. sieht man die
	Polizei kurz hinterherfahren/-laufen bei misslungenen Stunts)
	\item Charakter hat gerade Schule/Studium abgeschlossen und möchte eine
	Weltreise machen. Immer dabei: sein Skateboard.
	\item  Charakter hatte ein misslungenes Date. Jetzt ist er mit seinem Skateboard auf
	der Flucht vor dem Date (sieht man evtl. bei misslungenen Stunts durch Verlangsamung des
	Charakters kurz hinterherrennen)
\end{enumerate}
Der Charaktertyp wird auf die jeweilige Hintergrundgeschichte angepasst:

\begin{enumerate}
	\item Häftling ist eher der 'Haudrauf-Typ', macht Stunts sehr energisch, rastet kurz
aus wenn ein Stunt nicht klappt. (schimpft evtl. mit Spieler)
	\item Ist die ganze Zeit happy, misslungene Stunts machen ihm nicht wirklich etwas aus.
(ermutigt Spieler evtl. es ruhig nochmal zu versuchen)
	\item Total ängstlicher \& unsicherer Typ, schnell deprimiert bei vielen misslungenen
Stunts, große Angst vor verfolgendem Date.
\end{enumerate}

Hat der Charakter besondere Waffen o.ä.? Ja, Hintergrundgeschichte 1. \& 3. schon, da die
jeweiligen Charaktere ja auf der Flucht sind. Wenn sie einen Stunt nicht hinbekommen,
können sie evtl. mit einer Dose in Richtung des Verfolgers sprayen,
aus welcher dann eine Art Nebelwand herauskommt. Diese gibt dem Spieler genügend Zeit
um seine Geschwindigkeit wieder aufzuholen.
\clearpage

\section{Spielewelt:}
Als Orte werden vorerst Wälder, Straßen und Felder festgehalten. Auf den Feldern und
im Wald ist natürlich eine Art Weg, auf dem man gut fahren kann. Spieler lässt Charakter
beschleunigen, über Hindernisse springen und Stunts
ausführen. Bei Thematik 1. \& 3. soll eine eher hektische Stimmung erzeugt werden, da der
Spieler auf der Flucht ist. Bei Thematik 2. geht es einfach nur um den Spaß beim
Zocken und neue Tricks etc. zu erlernen.\\ \\
Die Hintergrundmusik soll etwas Hip-Hop-artiges bzw. rockiges sein. Die
verschiedenen Orte werden jeweils nach abfahren der einzelnen Level erkundet, je höher
das Level, desto schwieriger das Gebiet.\\ \\
Der Spieler bewegt sich größtenteils auf dem Skateboard fort, wenn er einen Stunt nicht
hinbekommt fällt er hin. Danach springt er direkt wieder auf sein Skateboard, um
schnell wieder beschleunigen und evtl. seinen Verfolgern entfliehen zu können.
Die Wälder und Felder sind im mystischen Stil gehalten, Spezial-Level welche kurzzeitig in
anderen Welten/auf anderen Planeten spielen sind auch möglich. Kurzes Auftauchen von
irgendwelchen mythologischen Wesen o.ä im Hintergrund und evtl. kurzes "Daumen-Hoch"
oder weiteres Anfeuern des Spielers als zukünftige Ideen.
\clearpage

\section{Spielerfahrung:}
Zu Beginn wird der Spieler in das Menü hineingeworfen. Hier sieht man groß das Logo 'hier
Spielename'. Im Hintergrund läuft eine kleine Animation des Spielhintergrundes. Im Menü
wird dem Spieler drei Optionen angeboten. Bei der Einstellungsauswahl sind
Lautstärkeregler. Beim Highscore werden die Bestplatzierten mit deren Punktezahl und
dem Rang des Spielers angezeigt. Und natürlich aktiviert man mit dem 'Spielen'-Button das
Game.\\ \\
Ohne viel Gerede wird man ins Spiel hineingeworfen. Der Hauptcharakter befindet sich in
einer von mehreren Skate-Maps. Man bekommt den Flair eines Oldschool Jump \& Runs.
Je nach Map-Thema bekommt man einen heiteren, aber hektischen Eindruck. Ziel ist es auch
das Level möglichst schnell durchzulaufen. Dafür suggeriert die Musik, Hintergrund und
Animationen, die einen aufhyped und sagen will, 'Auf dem Skateboard werde ich super
schnell'. Jeder Erfolg hat eine 'bessere' Animation, z.B gibt es für mehrere Punkte auf
Combos größere Zahlen und Soundeffekte.\\ \\
Die Kommandos, die der Spieler dem Charakter gibt, werden zeitnah ausgeführt,
um dem Spieler ein Erfolgserlebnis für erfolgreiche Stunts und Combos zu bieten. Dafür hilft
es darauf zu achten, wann die Animation der Stunts endet, damit man ein Gefühl dafür
bekommt, wann man sicher auf dem Boden ankommt.\\ \\
Die Stunt-Combos sollen eine Herausforderung bieten. Zum einen wird die
Reaktionsgeschwindigkeit beansprucht, als auch die antrainierte Musclememory. Ziel ist es
bei einer erfolgreichen Combo dem Spieler mit höheren Punkten und einem
Geschwindigkeitsboost zu belohnen.\\ \\
Was passiert, wenn man einen Stunt verfehlt und der Charakter auf den Boden fällt? Je
nach Level hat man ein Verfolger der einem im Nacken hängt. Dadurch soll eine
Anspannung entstehen, die immer größer wird je öfter man Zeit damit verschwendet auf den
Boden zu fallen.\\ \\
Spiel Ende: hat der Charakter das Ende der Map erreicht, wird einem angezeigt,
dass man das Level geschafft hat. Zusätzlich werden einem der Punktestand und die
absolvierte Zeit angegeben. Falls der Spieler es in die Top 10, 5 schafft oder Erster geworden ist,
wird ihm zusätzlich gratuliert.\\ \\
Die Punkten die man mit jedem Level einsammelt, kann man zum unlocken von anderen
Maps benutzen, oder auch für andere Skateboard- und Charakterskins ausgeben.
Das Spiel soll sowohl die Spieler ansprechen, die bei Jump \& Runs ihren Spaß haben und
die, die ein kleines kompetitives Spiel wollen und ihre Reaktions-Skills dazu zu nutzen, um
den Highscore zu knacken.
\clearpage

\section{Beat Chart:}
\begin{tabular}[h]{|l|l|}
	\hline
	Level & Gefängnis Ausbruch \\
	\hline
	Tageszeit & Nachts \\
	\hline
	Geschätzte Spielzeit & ~2-5 min.\\
	Umgebung & Gefängnis-Kulisse, Suchscheinwerfer und Helikopter im
	Hintergrund.
\\
	\hline
	Benutzte Mechaniken & Zusätzliche Punkte durch Stunts, Boost durch 'perfect landing'.\\
	\hline
	Benutzte Gefahren & Hinfallen, Verfolger \\
	\hline
	Musik & schneller Hiphop-Track, der etwas bedrohliches hat. \\
	\hline
\end{tabular}
\clearpage

\end{document}
